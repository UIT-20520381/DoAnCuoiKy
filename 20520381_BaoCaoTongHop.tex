\documentclass[a4paper,14pt]{extreport}
\usepackage[utf8]{vietnam}
\usepackage{array}
\usepackage{amsmath, amsthm, amssymb,latexsym,amscd,amsfonts,enumerate}

\usepackage[top=3.5cm, bottom=3.0cm, left=3.5cm, right=2.0cm]{geometry} 

\usepackage{color, fancyhdr, graphicx, wrapfig}

\usepackage[unicode]{hyperref}

\pagenumbering{roman}\pagestyle{plain}

\pagestyle{fancy}

\lhead{\it Kỹ năng nghề nghiệp}

\rfoot{\it SS004.M13}

\renewcommand{\headrulewidth}{1,2pt}

\renewcommand{\footrulewidth}{1,2pt}

\begin{document}

\fontsize{13pt}{18pt}\selectfont

\setlength{\baselineskip}{18truept}

\begin{titlepage} 

\begin{center}

{\large\bf TRƯỜNG ĐẠI HỌC CÔNG NGHỆ THÔNG TIN}\\


{———————o0o——————–}

\vskip 4cm

{\bf ĐỒ ÁN CUỐI KÌ}\\[1cm]

{\Large\bf \textbf{BÁO CÁO TỔNG HỢP ĐỒ ÁN CUỐI KÌ }}\\

\vskip 1cm



\vskip 5cm

\begin{tabular}{r l}

Giảng viên hướng dẫn:&{\bf NGUYỄN VĂN TOÀN}\\[0.5cm]

Sinh viên:&{\bf ĐỖ NGỌC QUANG ANH}\\[0.5cm]

Lớp:&{\bf SS004.M13}

\end{tabular}

\vfill

{\bf TP.HCM, 12/2021}

\end{center}

\end{titlepage}




\tableofcontents % Lệnh mục lục

\chapter{HỢP ĐỒNG NHÓM} % Chương 1

\pagenumbering{arabic}            % Danh lai trang

\section{Group name: Q.A}
\section{Danh sách thành viên trong nhóm:}
\begin{tabular}{|c|c|c|}
\hline
STT& TÊN & MSSV\\

\hline
1& Đỗ Ngọc Quang Anh& 20520381\\
\hline
\end{tabular}
\section{Nơi làm việc nhóm:}
\url{https://trello.com/b/OWBvCFxk/doancuoiky}\\
\url{https://github.com/UIT-20520381/DoAnCuoiKy}
\section{Mục tiêu thành lập nhóm:}
\begin{enumerate}[-]
\item  Cố gắng hoàn thành tốt công việc, đạt kết quả cao trong các bài tập, công việc được giao trong môn học.
\item Nâng cao kỹ năng làm việc độc lập cũng như các kỹ năng mềm khác.
\end{enumerate}
\section{ Hiệp định và phân công:}
\subsection { Hiệp định:}
\begin{enumerate}[-]
\item Hoàn thành công việc đúng thời hạn.
\item Đề cao sự tôn trọng  giữa các thành viên trong nhóm
\item Mục đích cuối cùng đạt được là hoàn thành tốt công việc nhóm.
\end{enumerate}

\subsection{ Bảng tiêu chí đánh giá:}
\begin{tabular}{| p{2.5cm}| p{3cm} |p{3cm} |p{3cm} |p{3cm}|}
     \hline
    Đặc điểm& Nổi bật& Tốt& Bình thường& Kém \\
    \hline
    Thái độ làm viêc& Sẵn sàng nhận nhiệm vụ và hoàn thành tốt nhiệm vụ.& Hoàn thành nhiệm vụ được giao.& Hoàn thành nhiệm vụ với sự nhắc nhở.& Không hoàn thành nhiệm vụ được giao.\\
    \hline
    Quản lý thời gian& Hoàn thành nhiệm vụ trước thời gian và đúng giờ trong các buổi họp nhóm.& Hoàn thành nhiệm vụ đúng thời gian và trể  không  quá 5 phút trong các buổi họp nhóm.& Hoàn thành nhiệm vụ đúng thời hạn với sự nhắc nhở và trể 5-10 phút trong các buổi họp nhóm.& Không hoàn thành nhiệm vụ và trễ quá 10 phút trong các buổi họp nhóm.\\
    \hline
    Giải quyết vấn đề phát sinh& Tích cực tìm kiếm giải pháp để giải quyết vấn đề phát sinh.& Nhờ người khác giải quyết vấn đề phát sinh.& Không giải quyết nnungw đưa ra ý kiến đóng góp.& Không tham gia vào vấn đề cần giải quyết.\\   
    \hline
    Nêu ý kiến& Sẳn sàng nêu ý kiến.& Chỉ nêu ý  kiến khi có việc cần.& Đưa ra ý kiến khi có sự nhắc nhở.& Không nêu ý kiến cho nhóm.\\
    \hline
    Giữ liên lạc& Luôn giữ liên lạc với nhóm.& Liên lạc trên 5 lần mới được.& Liên lạc từ 2-5 lần mới được.& Không liên lạc với nhóm.\\
    \hline
    
\end{tabular}
\subsection{ Phân công công việc:}
\begin{tabular}{|c | p{10cm}|}
    \hline 
    Đỗ Ngọc Quang Anh& \begin{enumerate}[-]
        \item Làm báo cáo.
        \item code game.
     \end{enumerate}\\
     \hline
\end{tabular}
\subsection{ Cam kết:}
\begin {enumerate}[-]
\item Sau khi đọc kỹ các nội dung mà hợp đồng nhóm nêu ra, các thành viên trong nhóm cam kết sẽ thực hiện đúng những yêu cầu đề ra.

\end{enumerate}

\begin{tabular}{p{10cm} p{4cm}}
Đỗ Ngọc Quang Anh
\end{tabular}
\newpage 

\chapter{GAME CON RẮN} % Chương 2


\section{Giới thiệu và hướng dẫn}
 \item Trong game này, người chơi sẽ dùng các phím W,S,A,D (lên,xuống,trái,phải) điều khiển con rắn đi ăn những quả táo, khi ăn thì con rắn sẽ ngày càng dài thêm. Người chơi sẽ bị thua khi đụng tường bao quanh hoặc con rắn tự cắn chính bản thân nó.
\section{Tài liệu kỹ thuật của trò chơi}
\begin{enumerate}[-]
\item struct Point() : Biểu diễn 1 điểm trên mặt phẳng 2D (x,y).
\item vector<Point> A : Thể hiện 1 dãy các điểm trên mặt phẳng 2D để lưu con rắn và khởi tạo con rắn.
\item class CONRAN : 
      \begin{enumerate}[+]
            \item void Ve() : Vẽ con rắn bằng cách dùng hàm gotoxy di chuyển con trỏ đến các điểm đã lưu và in ra "X" thể hiện đốt rắn.
            \item void DiChuyen(int Huong) : Phần di chuyển có 4 hướng (trái, lên, phải, xuống), ta sẽ quy định 4 hướng lần lượt là 0, 1, 2, 3 và một biến toàn cục (Huong) để xác định vị trí hiện tại của con rắn. Để làm con răng di chuyển, chúng ta sẽ dịch các phần tử về bên phải (con rắn đang bò theo phần đầu) và chừa chỗ trống cho phần đầu nhận giá trị mới. Giá trị tọa độ mới của phần đầu sẽ tùy theo hướng đang di chuyển. 
            \item void XoaDuoi() : Khi vẽ lại sự di chuyển thì phần đuôi cũ của con rắn không bị xóa, nên ta sẽ đặt 1 Point duoi dể lưu phần đuôi cũ của con rắn ở hàm DiChuyen(). Khi con rắn di chuyển thì phần đuôi cũ sẽ in ra " ". Ta không dùng lệnh system("cls") bởi vì nó sẽ xóa toàn màn hình, các hàm khác sẽ làm lại mới và như vậy thì phần tường sẽ bị nhấp nháy.
        \end{enumerate} 
\item void VeTuong() Để vẽ tường thì ta chỉ cần cho vòng lặp rồi in ra các ký tự thể hiện cho các bức tường, sau đó chúng ta sẽ gọi hàm vẽ này trong vòng lặp lúc nảy để nó vẽ lại tường mỗi lần con rắn di chuyển.
\item bool DungTuong() : Kiểm tra xem tọa độ của đầu rắn có trùng với tường hay không.
\item void KhoiTaoMoi() : Gán tọa độ x,y của con mồi ngẫu nhiên trong phạm vi của tường, sau khi có tọa độ mồi thì in ra màn hình.
\item bool AnMoi() : Kiểm tra xem tọa độ của đầu rắn có trùng với tọa độ của mồi hay không.
\item bool TuCanMinh() : Kiểm tra xem phần đầu rắn có trùng với các phần còn lại hay không. 
\item void BangDiem() : In ra số điểm.
\item void BatDau() : Gọi các hàm VeTuong(), KhoiTaoMoi(), BangDiem() để vẽ tường, tạo mồi, tạo bảng điểm. Khi con rắn di chuyển chúng ta sẽ vẽ lại chuyển động của nó. Do đó ta  sẽ dùng một vòng lặp, giữa các lần lặp sẽ nhận điều khiển từ bàn phím (w,A,S,D), di chuyển con rắn và vẽ lại. Kiểm tra các hàm bool như nếu rắn ăn được mồi thì điểm cộng lên 1 và tạo lại mồi, nếu rắn đụng tường hoặc tự cắn mình thì thua.
\item void XuLyThua() : Nếu thua thì người chơi có 2 lựa chọn, 1 là chơi lại, 2 là thoát game.
\end{enumerate}
\section{Mô tả quá trình làm việc}
\begin{enumerate}[-]
\item 30/11/2021: Làm hợp đồng nhóm tuần 2 và đăng lên GitHub.
\item 5/12/2021 - 12/12/2021: thực hiện code, fixbug.
\item 13/12/2021: Đăng code lên GitHub.
\item 14/12/2021: Làm báo tổng hợp và đăng lên GitHub.
\end{enumerate}
\section{Kỹ năng mà nhóm áp dụng}
\begin{enumerate}[-]
\item Kỹ năng làm việc độc lập.
\item Kỹ năng về code như con trỏ, lớp, hàm.
\item Tìm kiếm tài liệu.
\end{enumerate}

\chapter {ĐÁNH GIÁ}
\begin{enumerate}
    \item Đánh giá mức độ làm việc của thành viên trong nhóm.
    \begin{enumerate}[-]
        \item Đỗ Ngọc Quang Anh
        \begin{enumerate}[+]
            \item Thái độ làm việc: Tốt.
            \item Quản lý thời gian: Tốt.
            \item Giải quyết vấn đề phát sinh: Tốt.
        \end{enumerate}
    \end{enumerate}   
\end{enumerate}
\end{document}